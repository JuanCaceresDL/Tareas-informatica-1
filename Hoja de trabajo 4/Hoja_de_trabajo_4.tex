\documentclass{article}
\usepackage[utf8]{inputenc}

\title{Hoja de trabajo 4}
\author{Juan Esteban Cáceres de León}
\date{para 30 de agosto}

\usepackage{natbib}
\usepackage{graphicx}

\begin{document}

\maketitle

\section{Primera serie}
Serie de definiciones de conjuntos para $2^{\mathbb{N}}$ Indicar que definiciones corresponden al mismo conjunto.

\begin{enumerate}
    \item {$a:=\{1,2,4,8,16,32,64\}$} corresponde al conjunto de: {$d:=\{n\in\mathbb{N}\ |\ \exists i\in\mathbb{N}\ .\ n=2^i\wedge n<100 \}$}
    \item {$b:=\{n\ \in \mathbb{N}\ |\ \exists x \in \mathbb{N}\ .\ x=n/5 \}$} corresponde al conjunto de: {$f:=\{ n\in\mathbb{N}\ |\ \exists x\in \mathbb{N}\ .\ n=x+x+x+x+x \}$}
    \item {$c:=\{n\in \mathbb{N}\ |\ \exists x\in\mathbb{N}\ .\ n=x*x \}$} corresponde al conjunto de: {$e:=\{ n\in\mathbb{N}\ |\ \exists x\in \mathbb{N}\ .\ x=\sqrt{n} \}$}
\end{enumerate}
\section{Segunda serie}
Utilización de la jerga matemática:
\begin{enumerate}
    \item {El conjunto de todos los naturales divisibles dentro de $5$}
    \begin{center}
       A:= {$\{n\in \mathbb{N}\ |\ \frac{n}{5}\}$}
    \end{center}
    \item {El conjunto de todos los naturales divisibles dentro de $4$ y $5$}
    \begin{center}
       B:=  {$\{n\in \mathbb{N}\ |\ \frac{n}{5} \wedge \frac{n}{4}\}$}
    \end{center}
\item{El conjunto de todos los naturales que son primos}
\begin{center}
    $C:= \{ a|\forall 1<x<a. a mod(x)\not\equiv 0 \}$
\end{center}
\item {El conjunto de todos los conjuntos de numeros naturales que contienen
        un numero divisible dentro de $15$}
        \begin{center}
            $D:= \{a|a := \{ n\in \mathbb{N} | \exists x \in a| xmod(15)\equiv 0\} \}$
        \end{center}
    \item {El conjunto de todos los conjuntos de numeros naturales que al ser sumados
        producen $42$ como resultado}
        \begin{center}
            $E:= \{ a|a:= \{ n\in \mathbb{N} | \sum_{i=1}^{|A|} n_i = 42 \} \}$
        \end{center}
\end{enumerate}
\section{Tercera serie}

\begin{center}
    $S:= \{(a,b,c)| amod(2) \equiv 0, bmod(2) \equiv 0, a \not= b, c = a \otimes b, c \geq 30 \}$
\end{center}

\section{Cuarta serie}

\begin{enumerate}
    \item $F:=\{ f|f\in \mathbb{N}. (fmod2)\equiv 0 \}$
    \item $G:= \{ x|x \in \mathbb{N} \wedge b \in \mathbb{B}.\frac{x}{5} \equiv True \Leftrightarrow xmod(5) \equiv 0 \}$
   \begin{center}
       $Definamos: \\
    f:= \mathbb{R} \Rightarrow \mathbb{R}^+; f(x) = x^2$\\
    $g:= \mathbb{R} \Rightarrow \mathbb{R};g(x) = x$
    \end{center}
    \item $fog: \mathbb{R} \Rightarrow \mathbb{R}^2$
   Por tanto: $D:=\mathbb{{R}}$
   \item $fog: \mathbb{R} \Rightarrow \mathbb{R}^+$
   Por tanto: $R:= \mathbb{R}^+$
\end{enumerate}
 


\section{Quinta serie}
\begin{enumerate}
    \item {$f(x)=x^2$}
no es inyectiva porque hay mas de una X por Y, si es surjectiva, por lo tanto también implica que no sea biyectiva porque no se cumplen ambos aspectos.
\item {$g(x)=\frac{1}{cos(x-1)}$}

Existe más de un valor de equis para cada valor de y, por tanto no es inyectiva, el dominio son los reales pero el rango son $\mathbb{R}\neg[-1,1]$, por tanto tampoco es subyectiva ni biyectiva.
\item{$h(x)=2x$}
Existe un único valor de equis para cada "y" y el dominio y rango son los reales, por tanto es biyectiva, lo que implica subyectividad e inyectividad.

\item{$w(x)=x+1$}
Existe un único valor de equis para cada "y" y el dominio y rango son los reales, por tanto es biyectiva, lo que implica subyectividad e inyectividad.
\end{enumerate}
\section{Sexta serie}

\begin{enumerate}
    \item $B:= \{(a,b)| a=2b, b \in \mathbb{N}\}$
    \item $C:= \{(a,b)|a = 2b-1, b \in \mathbb{N}\}$
    \item $D:= \{(a,b)|a = 2(-b)-1, b \in \mathbb{Z}^- \}$
    \item $E:= \{(a,b)|a=f(b),b\in \mathbb{Z} \}$
\end{enumerate}
\begin{center}
    $f(x)= \{ 2x \Leftrightarrow x>0, \\
    0 \Leftrightarrow x = 0,\\
    2(-x)-1 \Leftrightarrow x<0\}$
\end{center}
\end{document}








