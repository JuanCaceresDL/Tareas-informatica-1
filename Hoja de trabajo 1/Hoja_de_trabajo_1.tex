\documentclass{article}
\usepackage[utf8]{inputenc}

\title{Hoja de trabajo 1}
\author{JuanCaceresDL}
\date{Julio 26 del 2018}

\usepackage{natbib}
\usepackage{graphicx}

\begin{document}

\maketitle

\section{Ejercicio 2}
\begin{enumerate}
\item El conjunto de nodos del grafo es: $\{1, 2, 3, 4, 5, 6\}$

\item {El conjunto de vertices del grafo son: 
$\newline\lbrace<1,2>,<2,6>,<2,3>,<5,1>\rbrace
\newline\lbrace<1,5>,<3,6>,<6,2>,<5,3>\rbrace
\newline\lbrace<1,3>,<2,4>,<3,5>,<5,4>\rbrace
\newline\lbrace<1,4>,<6,5>,<4,5>,<3,1>\rbrace
\newline\lbrace<3,2>,<5,6>,<6,4>,<2,1>\rbrace
\newline\lbrace<4,1>,<4,6>,<6,3>,<4,2>\rbrace$}

\end{enumerate}
\begin{figure}[h!]
\centering
\includegraphics[scale=0.3] {dados}


\end{figure}

\section{Ejercicio 3}
\begin{enumerate}
\item La estructura de datos que se podría usar es llamado "camino"

\item El algoritmo se basa en la estructura de camino para que no se limite la posibilidad de pasar por el mismo número que ya se ha pasado en el lanzamiento.

\item Para que el algoritmo sea efectivo y finito, no tiene que tener un ciclo.
\end{enumerate}
\end{document}
