\documentclass{article}
\usepackage[utf8]{inputenc}

\title{Hoja de trabajo 3}
\author{Juan Esteban Cáceres de León }

\usepackage{natbib}
\usepackage{graphicx}

\begin{document}

\maketitle

\section{Ejercicio 1}
\[
        n\oplus m := \left\{
        \begin{array}{l l}
            m & \mbox{si } n=o \\
            n & \mbox{si } m=o \\
            s(i\oplus m) & \mbox{si } n=s(i) \\
        \end{array}
        \right.
    \]
\begin{center}
    $s(s(s(0))) \oplus s(s(s(s(0))))$
    
\end{center}  
\begin{center}
    $s(s(s(s(0))) \oplus s(s(s(0))))$
    
\end{center}
\begin{center}
    $s(s(s(s(s(0))) \oplus s(s(0))))$
   
\end{center}
\begin{center}
    $s(s(s(s(s(s(0))) \oplus s(0))))$
    
\end{center}
\begin{center}
    $s(s(s(s(s(s(s(0))) \oplus 0))))$
   
\end{center}
\begin{center}
    $s(s(s(s(s(s(s(0 \oplus 0)))))))$
   
\end{center}
\begin{center}
    $s(s(s(s(s(s(s(0)))))))$
\end{center}
\section{Ejercicio 2}
\[
        n\otimes m := \left\{
        \begin{array}{l l}
            0 & \mbox{si } n=o \\
            0 & \mbox{si } m=o \\
            m & \mbox{si } n=1 \\
            n & \mbox{si } m=1 \\
            s(i)\otimes s(j) & \mbox{si } s(i) \oplus (s(i) \otimes j) \\
        \end{array}
        \right.
    \]

\section{Ejercicio 3}
\begin{enumerate}
    \item {$s(s(s(0)))\otimes 0$}
    \begin{center}
        $s(s(s(0))) \otimes 0 = 0$
    \end{center}
    \item {$s(s(s(0)))\otimes s(0)$}
    \begin{center}
        $s(s(s(0))) \otimes s(0) = s(s(s(0))) \oplus (s(s(s(0))) \otimes 0) = s(s(s(0)))$
    \end{center}
    \item {$s(s(s(0)))\otimes s(s(0))$}
   \begin{center}
       $s(s(s(0))) \oplus (s(s(s(0))) \otimes s(0))$
   \end{center}
   \begin{center}
       $s(s(s(0))) \oplus s(s(s(0)))$
   \end{center}
   \begin{center}
       $s(s(s(s(0))) \oplus s(s(0)))$
   \end{center}
   \begin{center}
       $s(s(s(s(s(0))) \oplus s(0)))$
   \end{center}
   \begin{center}
       $s(s(s(s(s(s(0))) \oplus 0)))$
      
   \end{center}
   \begin{center}
       $s(s(s(s(s(s(0 \oplus 0))))))$
   \end{center}
   \begin{center}
       $s(s(s(s(s(s(0))))))$
   \end{center}
\end{enumerate}
\section{Ejercicio 4}
\begin{enumerate}
    \item $a \oplus s(s(0)) = s(s(a))$
\begin{center}
    Caso base: $a = 0$
\end{center}
\begin{center}
    $0\oplus s(s(0)) = s(s(0))$
\end{center}
\begin{center}
    $s(s(0)) = s(s(0))$
\end{center}
\begin{center}
    Caso inductivo: $a = s(i)$
\end{center}
\begin{center}
    $s(i) \oplus s(s(0)) = s(s(s(i)))$
\end{center}
\begin{center}
    $s(s(s(i \oplus 0))) = s(s(s(i)))$
\end{center}
\begin{center}
    $s(s(s(i))) = s(s(s(i)))$
\end{center}
    \item $a \otimes b = b \otimes a$
\begin{center}
    caso base: $a = 0$
\end{center}
\begin{center}
    $0 \otimes b = b \otimes 0$
\end{center}
\begin{center}
    $0 = 0$
\end{center}
\begin{center}
    Caso inductivo: $a = s(i)$
\end{center}
\begin{center}
    $s(i) \otimes b = b \otimes s(i)$
\end{center}
\begin{center}
    $s(i) \oplus (s(i) \otimes b) = s(i) \oplus (s(i) \otimes b)$
\end{center}
\begin{center}
    $s(i) \otimes b = s(i) \otimes b$
\end{center}
    \item $a \otimes (b \otimes c) = (a \otimes b) \otimes c$
\begin{center}
    Caso base: $a = 0$
\end{center}
\begin{center}
    $0 \otimes (b \otimes c) = (0 \otimes b) \otimes c$
\end{center}
\begin{center}
    $0 \otimes (bc) = (0) \otimes c$
\end{center}
\begin{center}
    $0 = 0$
\end{center}
\begin{center}
    Caso inductivo: $a = s(i)$
\end{center}
\begin{center}
    $s(i) \otimes (b \otimes c) = (s(i) \otimes b) \otimes c$
\end{center}
\begin{center}
    $s(i) \oplus (s(i) \otimes (b \otimes c)) = (s(i) \oplus (s(i) \otimes b)) \otimes c$
\end{center}
\begin{center}
    $s(i) \oplus (s(i) \otimes (b \otimes c)) = s(i) \oplus (s(i) \otimes (b \otimes c))$
\end{center}
    \item $(a \otimes b) \otimes c = (a \otimes c) \oplus (b \otimes c)$
\end{enumerate}
\begin{center}
    Caso base: $c = 0$
\end{center}
\begin{center}
    $(a \otimes b) \otimes 0 = (a \otimes 0) \oplus (b \otimes 0)$
\end{center}
\begin{center}
    $(ab) \otimes 0 = (0) \oplus (0)$
\end{center}
\begin{center}
    $0 = 0$
\end{center}
\begin{center}
    Método inductivo: $c = n \oplus 1$
\end{center}
\begin{center}
    $(a \otimes b) \otimes (n \oplus 1) = (a \otimes (n + 1)) \oplus (b \otimes (n \oplus 1))$
\end{center}
\begin{center}
    $(a \otimes (n \oplus 1) \oplus (b \otimes (n \oplus 1)) = (an \oplus a) \oplus (bn \oplus b)$
\end{center}
\begin{center}
    $(an \oplus a) \oplus (bn \oplus b) = (an \oplus a) \oplus (bn \oplus b)$
\end{center}
\begin{center}
    $an \ominus an \oplus bn \ominus bn \oplus a \ominus a \oplus b \ominus b = 0$
\end{center}
\begin{center}
    $0 = 0$
\end{center}
\end{document}
